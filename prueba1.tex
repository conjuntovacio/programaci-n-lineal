
\documentclass{article}
\usepackage[utf8]{inputenc}
\usepackage{amsmath}
\usepackage[spanish]{babel}

\title{Apuntes de programación lineal}

\author{Melissa Alemán}


\begin{document}



\maketitle
\tableofcontents
\section{Introducción}
\label{sec:introduccion}

La forma estándar de un problema de programación lineal es: Dados una
matriz $A$ y vectores $b,c$, maximizar $c^Tx$ sujeto a $Ax\leq b$.

La forma simplex de un problema de programación lineal es: Dados una
matriz $A$ y vectores $b,c$, maximizar $c^tx$ sijeto a $Ax = b$.

\bigskip

\section{Ejercicios}
\label{sec:ejercicios}


Ejercicios :

1. considere el siguiente problema:
Maximizar $x+y$ sujeto a: $x \geq 0$, $1 \leq y \leq 3$, $2x+y \leq
7$.

2. Un gerente esta pleneando como distribuir la produccion de dos
productos entre dos maquinas. Para ser manufacturado cada producto

requiere cierto tiempo (en horas) en cada una de las maquinas. El
tiempo requerido está resumido en la siguiente tabla:

\begin{tabular}{|c|c|c|}
  \hline
  &A&B\\
  \hline
  Maquina 1&1&2\\
  maquina 2&1&1\\
  \hline
\end{tabular}

considere la siguiente matriz:

\begin{equation}
  \label{eq:1}
  A=\begin{pmatrix}
    0&1&2\\
    3&-1&5\\
  \end{pmatrix}
  \begin{pmatrix}
    2&0\\
    7&1\\
  \end{pmatrix}
\end{equation}
\end{document}
