\documentclass{article}

\usepackage[spanish]{babel}
\usepackage{amsmath}
\usepackage[utf8]{inputenc}

\title{Método Simplex}
\author{Melissa Alemán}

\begin{document}

\maketitle

\section{Introducción}
\label{sec:introduccion}


El método smplex es un algoritmo para resolver problemas e
programación lineal. Fue inventado por George Bernard Dantzig en el
año 1947.

\section{Ejemplo}
\label{sec:ejemplo}

Ilustraremos la alicacion del método simplex con un ejemplo.

Considere el siguiente problema:
\begin{equation*}
 \begin{aligned}
\text{Maximizar} \quad & 2x_1+2x_2\\
\text{sujeto a} \quad &
  \begin{aligned}
    2x_1 + x_2 &\leq 4\\
    -x_1 - 2x_2 &\geq -5\\
     x_1,x_2 &\geq  0
  \end{aligned}
\end{aligned}
\end{equation*}

como en una de las desigualdades aparecen las variables del lado
izquierdo con simbolo $\geq$ se multiplica la desugualdad por un -1 y
el símbolo cambia a $\leq$ y el ejercicio queda de la siguiente
manera:

\begin{equation*}
 \begin{aligned}
\text{Maximizar} \quad & 2x_1+2x_2\\
\text{sujeto a} \quad &
  \begin{aligned}
    2x_1 + x_2 + x_3 &\leq 4\\
    x_1 + 2x_2 + x_4 &\leq 5\\
     x_1,x_2 &\geq  0
  \end{aligned}
\end{aligned}
\end{equation*}

Para obtenerla forma simplex, añadimos una variable de holgura por
cada desigualdad.

\begin{equation*}
 \begin{aligned}
\text{Maximizar} \quad & 2x_1+2x_2\\
\text{sujeto a} \quad &
  \begin{aligned}
    2x_1 + x_2 + x_3\phantom{+x_4} &= 4\\
    x_1 + 2x_2\phantom{+x_3}+x_4 &= 5\\
     x_1,x_2, x_3, x_4 &\geq  0
  \end{aligned}
\end{aligned}
\end{equation*}

A continuacion obtenemos un \emph{tablero simplex} despejando las
variables de holgura

\begin{equation*}
 \begin{aligned}
    x_3 &= 4 - 2x_1 - x_2 \\
    x_4 &= 5 - x_1 - 2x_2\\
    \hline
    z &=\phantom{-1}2x_1+2x_2
\end{aligned}
\end{equation*}

\end{document}
